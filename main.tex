\documentclass[10pt,twoside,slovak,a4paper]{article}

\usepackage[slovak]{babel}
\usepackage[IL2]{fontenc} 
\usepackage[utf8]{inputenc}
\usepackage{graphicx}
\usepackage{url} 
\usepackage{hyperref} 

\usepackage{cite}

\pagestyle{headings}

\title{Názov\thanks{Semestrálny projekt v predmete Metódy inžinierskej práce, ak. rok 2022/2023, vedenie: Ing. Igor Stupavský}} 

\author{Matúš Remeň\\[2pt]
	{\small Slovenská technická univerzita v Bratislave}\\
	{\small Fakulta informatiky a informačných technológií}\\
	{\small \texttt{xremen@stuba.sk}}
	}

\date{\small              } 



\begin{document}

\maketitle

\begin{abstract}
\ldots
\end{abstract}



\section{Úvod}

V mojom článku sa budem snažiť porovnať vývoj mobilných a počítačových hier. Zameriam sa na rozdiely a podobnosti hlavne z technickej, ale napríklad aj z finančnej alebo marketingovej stránky. Prečo firmy vvvíjajú viac počítačových hier ako mobilných? Prečo hernému svetu vládnu počítačové hry? Aj tieto otázky sa budem snažiť zodpovedať v tomto článku.

Táto téma mi napadla, pretože donedávna som hral hry iba na mobile, ale teraz som začal aj s počitačovými hrami. Hneď som si všimol rozdiel v ponuke hier. Veľmi ma to zaujalo  a myslím si, že by to mohlo zaujať aj Vás.

\section{História vývoja hier} \label{historia} 

\subsection{Bertie the Brain} \label{historia:bertie}
Zrod videohier je sporná téma. Ľudia sa nedokážu zhodnúť na tom, ktorá aplikácia sa dá považovať za prvú videohru. Najstarší projekt, ktorý ľudia považujú za videohru sa nazýva Bertie the Brain. Bertie the Brain bol vytvorený Josfom Katesom v roku 1950. bol skôr počítač vysoký 4 metre, na ktorom sa dala hrať známa hra Tic-Tac-Toe alebo po slovensky Piškvorky.

\subsection{Tennis for Two} \label{historia:tennis}
Ako som spomenul vyššie, nie všetci považujú za prvú videohru to isté, a práve Bertie the Brain nemá veľa zástancov. Posúvame sa teda na ďaľšieho kandidáta, ktorý sa zároveň stáva aj víťazom. Je ním Tennis for Two. Tennis for Two bol vytvorený americkým fyzikom Williamom Higinbothamom v roku 1958. Túto hru môžeme považovať za predchodcu známej hru Pong.

\section{Nejaká časť} \label{nejaka}

Z obr.~\ref{f:rozhod} je všetko jasné. 

\begin{figure*}[tbh]
\centering
%\includegraphics[scale=1.0]{diagram.pdf}
Aj text môže byť prezentovaný ako obrázok. Stane sa z neho označný plávajúci objekt. Po vytvorení diagramu zrušte znak \texttt{\%} pred príkazom \verb|\includegraphics| označte tento riadok ako komentár (tiež pomocou znaku \texttt{\%}).
\caption{Rozhodujúci argument.}
\label{f:rozhod}
\end{figure*}



\section{Iná časť} \label{ina}

Základným problémom je teda\ldots{} Najprv sa pozrieme na nejaké vysvetlenie (časť~\ref{ina:nejake}), a potom na ešte nejaké (časť~\ref{ina:nejake}).\footnote{Niekedy môžete potrebovať aj poznámku pod čiarou.}

Môže sa zdať, že problém vlastne nejestvuje\cite{Coplien:MPD}, ale bolo dokázané, že to tak nie je~\cite{Czarnecki:Staged, Czarnecki:Progress}. Napriek tomu, aj dnes na webe narazíme na všelijaké pochybné názory\cite{PLP-Framework}. Dôležité veci možno \emph{zdôrazniť kurzívou}.


\subsection{Nejaké vysvetlenie} \label{ina:nejake}

Niekedy treba uviesť zoznam:

\begin{itemize}
\item jedna vec
\item druhá vec
	\begin{itemize}
	\item x
	\item y
	\end{itemize}
\end{itemize}

Ten istý zoznam, len číslovaný:

\begin{enumerate}
\item jedna vec
\item druhá vec
	\begin{enumerate}
	\item x
	\item y
	\end{enumerate}
\end{enumerate}


\subsection{Ešte nejaké vysvetlenie} \label{ina:este}

\paragraph{Veľmi dôležitá poznámka.}
Niekedy je potrebné nadpisom označiť odsek. Text pokračuje hneď za nadpisom.



\section{Dôležitá časť} \label{dolezita}




\section{Ešte dôležitejšia časť} \label{dolezitejsia}




\section{Záver} \label{zaver} % prípadne iný variant názvu



%\acknowledgement{Ak niekomu chcete poďakovať\ldots}


% týmto sa generuje zoznam literatúry z obsahu súboru literatura.bib podľa toho, na čo sa v článku odkazujete
\bibliography{literatura}
\bibliographystyle{alpha} % prípadne alpha, abbrv alebo hociktorý iný
\end{document}
