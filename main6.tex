\documentclass[10pt,twoside,slovak,a4paper]{article}

\usepackage[slovak]{babel}
\usepackage[IL2]{fontenc} 
\usepackage[utf8]{inputenc}
\usepackage{graphicx}
\usepackage{url} 
\usepackage{hyperref} 

\usepackage{cite}

\pagestyle{headings}

\title{Porovnanie vývoja počítačových a mobilných hier}

\author{Matúš Remeň\\[2pt]
	{\small Slovenská technická univerzita v Bratislave}\\
	{\small Fakulta informatiky a informačných technológií}\\
	{\small \texttt{xremen@stuba.sk}}
	}

\date{\small              } 



\begin{document}

\maketitle

\begin{abstract}
V mojom článku sa budem snažiť porovnať vývoj mobilných a počítačových hier. Zameriam sa na rozdiely a podobnosti hlavne z technickej, ale napríklad aj z finančnej alebo marketingovej stránky. Prečo firmy vvvíjajú viac počítačových hier ako mobilných? Prečo hernému svetu vládnu počítačové hry? Aj tieto otázky sa budem snažiť zodpovedať v tomto článku.

Táto téma mi napadla, pretože donedávna som hral hry iba na mobile, ale teraz som začal aj s počitačovými hrami. Hneď som si všimol rozdiel v ponuke hier. Veľmi ma to zaujalo  a myslím si, že by to mohlo zaujať aj Vás.
\end{abstract}



\section{Úvod}

Premýšľali ste niekedy nad tým, prečo viac hráčov preferuje počítačové hry pred tými mobilnými? Alebo nad tým, prečo spoločnosti venujú väčšiu pozornosť vývoju počítačoých hier? Presne nad týmito otázkami som sa začal zamýšľať aj ja. Najprv sú opísané najdôležitejšie míľniky vývoja videohier v časti ~\ref{historia}, potom je v šasti ~\ref{pocitacove} vysvetlený proces vývoja videohier, ďalej, v časti ~\ref{mobillne} je popísaný vývoj mobilných hier, v predposlednej časti ~\ref{porovnanie} je, ako môže názov napovedať, záverečné porovnanie vývoja hier pre tieto dve platformy a záverečné poznámky prináša časť ~\ref{zaver}.
\section{História vývoja hier} \label{historia} 

\subsection{Bertie the Brain} \label{historia:bertie}
Zrod videohier je sporná téma. Ľudia sa nedokážu zhodnúť na tom, ktorá aplikácia sa dá považovať za prvú videohru. Najstarší projekt, ktorý ľudia považujú za videohru sa nazýva Bertie the Brain. Bertie the Brain bol vytvorený Josfom Katesom v roku 1950. bol skôr počítač vysoký 4 metre, na ktorom sa dala hrať známa hra Tic-Tac-Toe alebo po slovensky Piškvorky.

\subsection{Tennis for Two} \label{historia:tennis}
Ako som spomenul vyššie, nie všetci považujú za prvú videohru to isté, a práve Bertie the Brain nemá veľa zástancov. Posúvame sa teda na ďaľšieho kandidáta, ktorý sa zároveň stáva aj víťazom. Je ním Tennis for Two. Tennis for Two bol vytvorený americkým fyzikom Williamom Higinbothamom v roku 1958. Túto hru môžeme považovať za predchodcu známej hry Pong.

\subsection{Battlezone} \label{historia:battlezone}
Veľká revolúcia pre herný svet nastala v roku 1980, keď uzrela svetlo sveta hra Battlezone. Veľká revolúcia to bola z jedného dôvodu, a tým je, že to bola prvá 3D hra. Hráči sa už nemuseli dívať ako sa objekty hýbu len po 2 osiach, ale konečne sa hry začali približovať ku skutočnosti.
\section{Ako vznikajú hry?} \label{pocitacove}

\subsection{Zrod hry}

Prvá vec, ktorú autor hry potrebuje je nápad. V dnešnej dobe existuje obrovské množstvo rôznych hier, takže aby bola hra úspešná potrebuje autor vytvoriť hru, ktorá bude niečím výnimočná. Potom prichádza čas, kedy musia programátori hru naprogramovať. Najideálnejšie je, ak hra neobsahuje žiadne tzv. "bugy", no niekedy to nemusí byť na škodu. Ďaľšia vec, ktorú potrebuje autor realizovať je animácia a celkovo estetická stránka. Veľmi dôležitá je aj hudba. Poďme si to ale opísať podrobenjšie.

\subsection{Plánovanie}
Prvým krokom je plánovanie, ktoré má viac častí. V prvej časti hľadá autor odpovede na otázky ako napríklad: Aký typ hry chcem vytvoriť? Bude to 2D alebo 3D? Aké budú kľúčové funkcie mojej hry? Aké v nej budú postavy? V druhej časti musí autor svoj nápad obhájiť pred herným štúdiom. V tejto časti sa hľadajú odpovede na otázky ako napríklad: Aké budú približne náklady na vývoj tejto hry? Máme na jej vývoj dostatočné technické schopnosti? Aký veľký bude tím vývojárov? Za ako dlho hru vydáme?

\subsection{Predprodukcia}
V druhom kroku sa premýšľa o tom ako vdýchnuť život nápadom z kroku 1. Scenáristi sa stretnú s autormi a predebatujú ako presne budú postavy vyzerať, aké budú ich životné príbehy, ako spolu súvisia. Programátori predebatujú so scenáristami technické aspekty, napríklad či sa niektoré časti hry dajú alebo nedajú naprogramovať. Výtvarníci kontrolujú, či sú farby, tiene, osvetlenie dôsledné. 


%Keď už má autor nápad, prvou dôležitou technickou zložkou je softvérové inžinierstvo. V skratke to znamená vytvorenie softvérovej časti hry, čiže jej naprogramovanie. Pri sofistikovanosti a cenách moderných hier je potrebné využívať čo najviac prvkov v čo najviac projektoch. Maximálne, ako sa dajú využiť sa využívajú tzv. Rozhrania programovania aplikácií (v angličtine API), ako napríklad Microsoft XNA alebo Nvidia PhysX. Softvérové inžinierstvo by sa dalo zhrnúť vetou: "Také jednoduché ako je možné, také komplexné ako je potrebné."

%Druhou podstatnou technickou zložkou je grafika. Vo všeobecnosti platí pravidlo, že čím vyzerá hra realistickejšie, tým lepšie, no stále sa nájdu aj fanúšikovia starej grafiky. Vytvorenie dobrej grafiky je taktiež veľmi komplikovaný proces, pretože tvorcovia musia odladiť všetky 3D modely, osvetlenie, tiene, animácie...~\cite{1} 




\section{Mobilné hry} \label{mobilne}

\subsection{Zrod mobilných hier}

Asi každému, kto vie niečo málo o technológiách musí byť jasné, že mobilné hry vznikli neskôr ako počítačové. Je to z jediného jednoduchého dôvodu, a ním je, že mobilné telefóny vznikli neskôr ako počítače. Prvá hra, ktorú si ľudia mohli zahrať na mobilnom telefóne zvnikla v roku 1997 (o viac ako 40 rokov neskôr ako prvá počítačoví hra) a bola to legendárna hra Snake dostupná na telefóne Nokia 6110.~\cite{2}

\subsection{Vývoj mobilných hier}
V podstate môžeme povedať, že proces vývoja mobilmých hier je taký istý ako pri počítačových hrách. Tak isto to celé musí začať nápadom, potom treba hru naprogramovať ... V čom je ale rozdiel, je hardvérová časť. Mobily majú menší výkon ako počítače, majú menšiu pamäť a samozrejme väčšina z nich nemá klávesnicu a ani myš, takže ovládanie je ďaľší rozdiel.

\section{Porovnanie} \label{porovnanie}

Povedal by som, že dôvodov, prečo sa má vývoj počítačových hier väčšiu pozornosť je hneď niekoľko. Prvým je historický. Počítače tu boli skôr ako mobily, teda aj počítačové hry vznikali skôr, a tým pádom aj prostriedky na ich vývoj vznikali skôr, kdežto pri mobilných hrách tieto prostriedky ešte nie sú až tak vyvinuté. Ďaľším dôvodom je, že počítač prináša omnoho kvalitnejší herný zážitok, preto viac ľudí volí počítače, a preto je po počítačových hrách väčší dopyt.

\section{Záver} \label{zaver}
napisem neskor

% týmto sa generuje zoznam literatúry z obsahu súboru literatura.bib podľa toho, na čo sa v článku odkazujete
\bibliography{literatura}
\bibliographystyle{alpha} % prípadne alpha, abbrv alebo hociktorý iný
\end{document}
